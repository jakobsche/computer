% This file has been created automatically by FPDoc.
% Linear output (c) 2005 Michael Van Canneyt
%%%%%%%%%%%%%%%%%%%%%%%%%%%%%%%%%%%%%%%%%%%%%%%%%%%%%%%%%%%%%%%%%%%%%%%
%%%%%%%%%%%%%%%%%%%%%%%%%%%%%%%%%%%%%%%%%%%%%%%%%%%%%%%%%%%%%%%%%%%%%%%
% Reference for unit 'ChipTemp'
%%%%%%%%%%%%%%%%%%%%%%%%%%%%%%%%%%%%%%%%%%%%%%%%%%%%%%%%%%%%%%%%%%%%%%%
\chapter{Reference for unit 'ChipTemp'}
\label{computer:chiptemp}
%%%%%%%%%%%%%%%%%%%%%%%%%%%%%%%%%%%%%%%%%%%%%%%%%%%%%%%%%%%%%%%%%%%%%%%
% Used units
\section{Used units}
\begin{FPCltable}{lr}{Used units by unit 'ChipTemp'}{ChipTemp:0units}
Name & Page \\ \hline
Classes\index{unit!Classes} & \pageref{computer:chiptemp:classes} \\
System\index{unit!System} & \pageref{computer:chiptemp:system} \\
SysUtils\index{unit!SysUtils} & \pageref{computer:chiptemp:sysutils} \\
\end{FPCltable}
%%%%%%%%%%%%%%%%%%%%%%%%%%%%%%%%%%%%%%%%%%%%%%%%%%%%%%%%%%%%%%%%%%%%%%%
% Overview
\section{Overview}
Die Unit enthält Komponenten, die über das sysfs-Kernelinterface Daten von Temperatursensoren abfragt. Insbesondere kann man darüber die Temperatur und Informationen über Temperaturschaltpunkte als Sensoreigenschaften ermitteln.%%%%%%%%%%%%%%%%%%%%%%%%%%%%%%%%%%%%%%%%%%%%%%%%%%%%%%%%%%%%%%%%%%%%%%%
% Procedures and functions
\section{Procedures and functions}
\label{chiptempfunctions}
% Register
\subsection{Register}
\label{computer:chiptemp:register}
\index{Register}
\begin{FPCList}
\Declaration 

\begin{verbatim}
procedure Register
\end{verbatim}
\Visibility
default
\end{FPCList}
%%%%%%%%%%%%%%%%%%%%%%%%%%%%%%%%%%%%%%%%%%%%%%%%%%%%%%%%%%%%%%%%%%%%%%%
% TChipTempSensor
\section{TChipTempSensor}
\label{computer:chiptemp:tchiptempsensor}
\index{TChipTempSensor}
% Description
\subsection{Description}
Mißt die aktuelle Temperatur und zeigt Eigenschaften von Schaltpunkten an% Method overview
\subsection{Method overview}
\label{computer:chiptemp:tchiptempsensor:methods}
\begin{tabularx}{\textwidth}{llX}
Page & Method & Description  \\ \hline
\pageref{computer:chiptemp:tchiptempsensor:destroy} & Destroy  &  \\
\hline
\end{tabularx}
% Property overview
\subsection{Property overview}
\label{computer:chiptemp:tchiptempsensor:properties}
\begin{tabularx}{\textwidth}{lllX}
Page & Properties & Access & Description \\ \hline
\pageref{computer:chiptemp:tchiptempsensor:id} & Id & rw & eindeutiger Sensorindex im System \\
\pageref{computer:chiptemp:tchiptempsensor:sensortype} & SensorType & r & Sensortyp \\
\pageref{computer:chiptemp:tchiptempsensor:temperature} & Temperature & r & gemessene Temperatur des Sensors in °C \\
\pageref{computer:chiptemp:tchiptempsensor:trippointdescriptions} & TripPointDescriptions & r & Beschreibung der Auslösepunkte \\
\pageref{computer:chiptemp:tchiptempsensor:trippoints} & TripPoints & r & Temperaturschaltpunkte \\
\hline
\end{tabularx}
% TChipTempSensor.Destroy
\subsection{TChipTempSensor.Destroy}
\label{computer:chiptemp:tchiptempsensor:destroy}
\index{TChipTempSensor.Destroy}
\begin{FPCList}
\Declaration 

\begin{verbatim}
destructor Destroy;  Override
\end{verbatim}
\Visibility
public
\end{FPCList}
% TChipTempSensor.SensorType
\subsection{TChipTempSensor.SensorType}
\label{computer:chiptemp:tchiptempsensor:sensortype}
\index{TChipTempSensor.SensorType}
\begin{FPCList}
\Synopsis
Sensortyp\Declaration 

\begin{verbatim}
Property SensorType : string
\end{verbatim}
\Visibility
public
\Access
Read
\Description
Die Eigenschaft ist ein kurzer Text, dessen Syntax und Bedeutung vom Kernel oder vom Hardwaretreiber des Sensors bestimmt wird. Sie kann Auskunft über den Ort des Sensors geben, z.B., indem die Bezeichnung des Hardwaremoduls angegeben ist.\end{FPCList}
% TChipTempSensor.Temperature
\subsection{TChipTempSensor.Temperature}
\label{computer:chiptemp:tchiptempsensor:temperature}
\index{TChipTempSensor.Temperature}
\begin{FPCList}
\Synopsis
gemessene Temperatur des Sensors in °C\Declaration 

\begin{verbatim}
Property Temperature : Extended
\end{verbatim}
\Visibility
public
\Access
Read
\Description
vom Sensor augenblicklich gemessene Temperatur in °C\end{FPCList}
% TChipTempSensor.TripPointDescriptions
\subsection{TChipTempSensor.TripPointDescriptions}
\label{computer:chiptemp:tchiptempsensor:trippointdescriptions}
\index{TChipTempSensor.TripPointDescriptions}
\begin{FPCList}
\Synopsis
Beschreibung der Auslösepunkte\Declaration 

\begin{verbatim}
Property TripPointDescriptions : TStrings
\end{verbatim}
\Visibility
public
\Access
Read
\Description
Die Liste enthält in jedeer Zeile einen Namen und einen Wert, getrennt durch einen Doppelpunkt, sofern solche Auslösepunkte programmiert sind.\end{FPCList}
% TChipTempSensor.TripPoints
\subsection{TChipTempSensor.TripPoints}
\label{computer:chiptemp:tchiptempsensor:trippoints}
\index{TChipTempSensor.TripPoints}
\begin{FPCList}
\Synopsis
Temperaturschaltpunkte\Declaration 

\begin{verbatim}
Property TripPoints[I: Integer]: TTripPoint
\end{verbatim}
\Visibility
public
\Access
Read
\Description
Ein gültiger Index ergibt einen gültigen Schaltpunkt (Reaktion des Kernels auf Erreichen der Schaltpunkttemperatur). Seine Bedeutung ergibt sich aus den weiteren Eigenschaften der TTripPoint-Komponente mit dem angegebenen Index\Errors
Ein ungültiger Index der TTripPoint-Komponente löst Exceptions aus, wenn andere Eigenschaften des Schaltpunkts abgefragt werden\end{FPCList}
% TChipTempSensor.Id
\subsection{TChipTempSensor.Id}
\label{computer:chiptemp:tchiptempsensor:id}
\index{TChipTempSensor.Id}
\begin{FPCList}
\Synopsis
eindeutiger Sensorindex im System\Declaration 

\begin{verbatim}
Property Id : Integer
\end{verbatim}
\Visibility
published
\Access
Read,Write
\Description
Das System numeriert die vorhandenen Temperatursensoren mit 0 beginnend durch. Wird eine gültige Nummer angegeben, enthalten die übrigen Eigenschaften Daten dieses Sensors.\end{FPCList}
%%%%%%%%%%%%%%%%%%%%%%%%%%%%%%%%%%%%%%%%%%%%%%%%%%%%%%%%%%%%%%%%%%%%%%%
% TChipTempSensors
\section{TChipTempSensors}
\label{computer:chiptemp:tchiptempsensors}
\index{TChipTempSensors}
% Description
\subsection{Description}
Für jeden Chiptemperatursensor, der im sysfs gefunden wird, enthält die Liste eine TChipTempSensor-Komponente.
\begin{FPCList}
\SeeAlso
TChipTempSensor (\pageref{computer:chiptemp:tchiptempsensor})\end{FPCList}
% Method overview
\subsection{Method overview}
\label{computer:chiptemp:tchiptempsensors:methods}
\begin{tabularx}{\textwidth}{llX}
Page & Method & Description  \\ \hline
\pageref{computer:chiptemp:tchiptempsensors:destroy} & Destroy  &  \\
\hline
\end{tabularx}
% Property overview
\subsection{Property overview}
\label{computer:chiptemp:tchiptempsensors:properties}
\begin{tabularx}{\textwidth}{lllX}
Page & Properties & Access & Description \\ \hline
\pageref{computer:chiptemp:tchiptempsensors:sensorcount} & SensorCount & r & Anzahl der Temperatursensoren \\
\pageref{computer:chiptemp:tchiptempsensors:sensors} & Sensors & r & Liste der Sensoren \\
\hline
\end{tabularx}
% TChipTempSensors.Destroy
\subsection{TChipTempSensors.Destroy}
\label{computer:chiptemp:tchiptempsensors:destroy}
\index{TChipTempSensors.Destroy}
\begin{FPCList}
\Declaration 

\begin{verbatim}
destructor Destroy;  Override
\end{verbatim}
\Visibility
public
\end{FPCList}
% TChipTempSensors.SensorCount
\subsection{TChipTempSensors.SensorCount}
\label{computer:chiptemp:tchiptempsensors:sensorcount}
\index{TChipTempSensors.SensorCount}
\begin{FPCList}
\Synopsis
Anzahl der Temperatursensoren\Declaration 

\begin{verbatim}
Property SensorCount : Integer
\end{verbatim}
\Visibility
public
\Access
Read
\Description
Anzahl der in der Eigenschaft Sensors verfügbaren Temperatursensoren\end{FPCList}
% TChipTempSensors.Sensors
\subsection{TChipTempSensors.Sensors}
\label{computer:chiptemp:tchiptempsensors:sensors}
\index{TChipTempSensors.Sensors}
\begin{FPCList}
\Synopsis
Liste der Sensoren\Declaration 

\begin{verbatim}
Property Sensors[I: Integer]: TChipTempSensor
\end{verbatim}
\Visibility
public
\Access
Read
\Description
Temperatursensoren im System\end{FPCList}
%%%%%%%%%%%%%%%%%%%%%%%%%%%%%%%%%%%%%%%%%%%%%%%%%%%%%%%%%%%%%%%%%%%%%%%
% TTripPoint
\section{TTripPoint}
\label{computer:chiptemp:ttrippoint}
\index{TTripPoint}
% Description
\subsection{Description}
Die Klasse zeigt Eigenschaften eines temperaturschaltpunkts an. Ihre Instanzen sind ggf. über die Eigenschaft TripPoints einer TChipTempSensor-Komponente ansprechbar.% Property overview
\subsection{Property overview}
\label{computer:chiptemp:ttrippoint:properties}
\begin{tabularx}{\textwidth}{lllX}
Page & Properties & Access & Description \\ \hline
\pageref{computer:chiptemp:ttrippoint:id} & Id & rw & Index des Auslösepunkts eines Sensors \\
\pageref{computer:chiptemp:ttrippoint:sensor} & Sensor & r & Sensor, zu dem der Schaltpunkt gehört \\
\pageref{computer:chiptemp:ttrippoint:temperature} & Temperature & r & Schalttemperatur in °C \\
\pageref{computer:chiptemp:ttrippoint:tptype} & TPType & r & Schaltpunkttyp \\
\hline
\end{tabularx}
% TTripPoint.Temperature
\subsection{TTripPoint.Temperature}
\label{computer:chiptemp:ttrippoint:temperature}
\index{TTripPoint.Temperature}
\begin{FPCList}
\Synopsis
Schalttemperatur in °C\Declaration 

\begin{verbatim}
Property Temperature : Extended
\end{verbatim}
\Visibility
public
\Access
Read
\Description
Temperatur, zu der der Schaltpunkt gehört\end{FPCList}
% TTripPoint.TPType
\subsection{TTripPoint.TPType}
\label{computer:chiptemp:ttrippoint:tptype}
\index{TTripPoint.TPType}
\begin{FPCList}
\Synopsis
Schaltpunkttyp\Declaration 

\begin{verbatim}
Property TPType : string
\end{verbatim}
\Visibility
public
\Access
Read
\Description
Kurze Beschreibung des Schaltpunkts. Die genaue Bedeutung ist in der Kerneldokumentation des Betriebssystems oder in der Treiberdokumentation des Sensors zu finden.\end{FPCList}
% TTripPoint.Sensor
\subsection{TTripPoint.Sensor}
\label{computer:chiptemp:ttrippoint:sensor}
\index{TTripPoint.Sensor}
\begin{FPCList}
\Synopsis
Sensor, zu dem der Schaltpunkt gehört\Declaration 

\begin{verbatim}
Property Sensor : TChipTempSensor
\end{verbatim}
\Visibility
public
\Access
Read
\Description
Der Besitzer der TTripPoint-Komponente muß eine TChipTempSensor-Komponente sein. Dann kann man mittels der Eigenschaft Sensor die Zuordnung des Schaltpunkts zu einem bestimmten Sensor feststellen.\end{FPCList}
% TTripPoint.Id
\subsection{TTripPoint.Id}
\label{computer:chiptemp:ttrippoint:id}
\index{TTripPoint.Id}
\begin{FPCList}
\Synopsis
Index des Auslösepunkts eines Sensors\Declaration 

\begin{verbatim}
Property Id : Integer
\end{verbatim}
\Visibility
published
\Access
Read,Write
\end{FPCList}
%%%%%%%%%%%%%%%%%%%%%%%%%%%%%%%%%%%%%%%%%%%%%%%%%%%%%%%%%%%%%%%%%%%%%%%
% TTripPointDescriptions
\section{TTripPointDescriptions}
\label{computer:chiptemp:ttrippointdescriptions}
\index{TTripPointDescriptions}
% Method overview
\subsection{Method overview}
\label{computer:chiptemp:ttrippointdescriptions:methods}
\begin{tabularx}{\textwidth}{llX}
Page & Method & Description  \\ \hline
\pageref{computer:chiptemp:ttrippointdescriptions:get} & Get  &  \\
\pageref{computer:chiptemp:ttrippointdescriptions:getcount} & GetCount  &  \\
\hline
\end{tabularx}
% TTripPointDescriptions.Get
\subsection{TTripPointDescriptions.Get}
\label{computer:chiptemp:ttrippointdescriptions:get}
\index{TTripPointDescriptions.Get}
\begin{FPCList}
\Declaration 

\begin{verbatim}
function Get(Index: Integer) : string;  Override
\end{verbatim}
\Visibility
protected
\end{FPCList}
% TTripPointDescriptions.GetCount
\subsection{TTripPointDescriptions.GetCount}
\label{computer:chiptemp:ttrippointdescriptions:getcount}
\index{TTripPointDescriptions.GetCount}
\begin{FPCList}
\Declaration 

\begin{verbatim}
function GetCount : Integer;  Override
\end{verbatim}
\Visibility
protected
\end{FPCList}
%%%%%%%%%%%%%%%%%%%%%%%%%%%%%%%%%%%%%%%%%%%%%%%%%%%%%%%%%%%%%%%%%%%%%%%
%%%%%%%%%%%%%%%%%%%%%%%%%%%%%%%%%%%%%%%%%%%%%%%%%%%%%%%%%%%%%%%%%%%%%%%
% Reference for unit 'computer'
%%%%%%%%%%%%%%%%%%%%%%%%%%%%%%%%%%%%%%%%%%%%%%%%%%%%%%%%%%%%%%%%%%%%%%%
\chapter{Reference for unit 'computer'}
\label{computer:computer}
%%%%%%%%%%%%%%%%%%%%%%%%%%%%%%%%%%%%%%%%%%%%%%%%%%%%%%%%%%%%%%%%%%%%%%%
% Used units
\section{Used units}
\begin{FPCltable}{lr}{Used units by unit 'computer'}{computer:0units}
Name & Page \\ \hline
ChipTemp\index{unit!ChipTemp} & \pageref{computer:chiptemp} \\
LazarusPackageIntf\index{unit!LazarusPackageIntf} & \pageref{computer:computer:lazaruspackageintf} \\
LEDView\index{unit!LEDView} & \pageref{computer:ledview} \\
Retain\index{unit!Retain} & \pageref{computer:retain} \\
SysFS\index{unit!SysFS} & \pageref{computer:sysfs} \\
System\index{unit!System} & \pageref{computer:computer:system} \\
\end{FPCltable}
%%%%%%%%%%%%%%%%%%%%%%%%%%%%%%%%%%%%%%%%%%%%%%%%%%%%%%%%%%%%%%%%%%%%%%%
%%%%%%%%%%%%%%%%%%%%%%%%%%%%%%%%%%%%%%%%%%%%%%%%%%%%%%%%%%%%%%%%%%%%%%%
% Reference for unit 'LEDView'
%%%%%%%%%%%%%%%%%%%%%%%%%%%%%%%%%%%%%%%%%%%%%%%%%%%%%%%%%%%%%%%%%%%%%%%
\chapter{Reference for unit 'LEDView'}
\label{computer:ledview}
%%%%%%%%%%%%%%%%%%%%%%%%%%%%%%%%%%%%%%%%%%%%%%%%%%%%%%%%%%%%%%%%%%%%%%%
% Used units
\section{Used units}
\begin{FPCltable}{lr}{Used units by unit 'LEDView'}{LEDView:0units}
Name & Page \\ \hline
Classes\index{unit!Classes} & \pageref{computer:ledview:classes} \\
Controls\index{unit!Controls} & \pageref{computer:ledview:controls} \\
Dialogs\index{unit!Dialogs} & \pageref{computer:ledview:dialogs} \\
ExtCtrls\index{unit!ExtCtrls} & \pageref{computer:ledview:extctrls} \\
Graphics\index{unit!Graphics} & \pageref{computer:ledview:graphics} \\
StdCtrls\index{unit!StdCtrls} & \pageref{computer:ledview:stdctrls} \\
System\index{unit!System} & \pageref{computer:ledview:system} \\
SysUtils\index{unit!SysUtils} & \pageref{computer:ledview:sysutils} \\
\end{FPCltable}
%%%%%%%%%%%%%%%%%%%%%%%%%%%%%%%%%%%%%%%%%%%%%%%%%%%%%%%%%%%%%%%%%%%%%%%
% Constants, types and variables
\section{Constants, types and variables}
\label{ledviewconststypesvars}
% Types
\subsection{Types}
\label{ledviewtypes}

\begin{verbatim}
TLabelPosition = (lpLeft,lpRight)
\end{verbatim}
\label{computer:ledview:tlabelposition}
\index{TLabelPosition}
\begin{FPCltable}{ll}{Enumeration values for type TLabelPosition
}{table1}
Value
 & Explanation
\\ \hline
lpLeft
 & \\
lpRight
 & \\
\end{FPCltable}


%%%%%%%%%%%%%%%%%%%%%%%%%%%%%%%%%%%%%%%%%%%%%%%%%%%%%%%%%%%%%%%%%%%%%%%
% Procedures and functions
\section{Procedures and functions}
\label{ledviewfunctions}
% Register
\subsection{Register}
\label{computer:ledview:register}
\index{Register}
\begin{FPCList}
\Declaration 

\begin{verbatim}
procedure Register
\end{verbatim}
\Visibility
default
\end{FPCList}
%%%%%%%%%%%%%%%%%%%%%%%%%%%%%%%%%%%%%%%%%%%%%%%%%%%%%%%%%%%%%%%%%%%%%%%
% TLEDView
\section{TLEDView}
\label{computer:ledview:tledview}
\index{TLEDView}
% Method overview
\subsection{Method overview}
\label{computer:ledview:tledview:methods}
\begin{tabularx}{\textwidth}{llX}
Page & Method & Description  \\ \hline
\pageref{computer:ledview:tledview:create} & Create  &  \\
\hline
\end{tabularx}
% Property overview
\subsection{Property overview}
\label{computer:ledview:tledview:properties}
\begin{tabularx}{\textwidth}{lllX}
Page & Properties & Access & Description \\ \hline
\pageref{computer:ledview:tledview:caption} & Caption & rw &  \\
\pageref{computer:ledview:tledview:labelposition} & LabelPosition & rw &  \\
\pageref{computer:ledview:tledview:ledoffcolor} & LEDOffColor & rw &  \\
\pageref{computer:ledview:tledview:ledoncolor} & LEDOnColor & rw &  \\
\pageref{computer:ledview:tledview:value} & Value & rw &  \\
\hline
\end{tabularx}
% TLEDView.Create
\subsection{TLEDView.Create}
\label{computer:ledview:tledview:create}
\index{TLEDView.Create}
\begin{FPCList}
\Declaration 

\begin{verbatim}
constructor Create(AOwner: TComponent);  Override
\end{verbatim}
\Visibility
public
\end{FPCList}
% TLEDView.Caption
\subsection{TLEDView.Caption}
\label{computer:ledview:tledview:caption}
\index{TLEDView.Caption}
\begin{FPCList}
\Declaration 

\begin{verbatim}
Property Caption : string
\end{verbatim}
\Visibility
published
\Access
Read,Write
\end{FPCList}
% TLEDView.LEDOffColor
\subsection{TLEDView.LEDOffColor}
\label{computer:ledview:tledview:ledoffcolor}
\index{TLEDView.LEDOffColor}
\begin{FPCList}
\Declaration 

\begin{verbatim}
Property LEDOffColor : TColor
\end{verbatim}
\Visibility
published
\Access
Read,Write
\end{FPCList}
% TLEDView.LEDOnColor
\subsection{TLEDView.LEDOnColor}
\label{computer:ledview:tledview:ledoncolor}
\index{TLEDView.LEDOnColor}
\begin{FPCList}
\Declaration 

\begin{verbatim}
Property LEDOnColor : TColor
\end{verbatim}
\Visibility
published
\Access
Read,Write
\end{FPCList}
% TLEDView.LabelPosition
\subsection{TLEDView.LabelPosition}
\label{computer:ledview:tledview:labelposition}
\index{TLEDView.LabelPosition}
\begin{FPCList}
\Declaration 

\begin{verbatim}
Property LabelPosition : TLabelPosition
\end{verbatim}
\Visibility
published
\Access
Read,Write
\end{FPCList}
% TLEDView.Value
\subsection{TLEDView.Value}
\label{computer:ledview:tledview:value}
\index{TLEDView.Value}
\begin{FPCList}
\Declaration 

\begin{verbatim}
Property Value : Boolean
\end{verbatim}
\Visibility
published
\Access
Read,Write
\end{FPCList}
%%%%%%%%%%%%%%%%%%%%%%%%%%%%%%%%%%%%%%%%%%%%%%%%%%%%%%%%%%%%%%%%%%%%%%%
%%%%%%%%%%%%%%%%%%%%%%%%%%%%%%%%%%%%%%%%%%%%%%%%%%%%%%%%%%%%%%%%%%%%%%%
% Reference for unit 'Retain'
%%%%%%%%%%%%%%%%%%%%%%%%%%%%%%%%%%%%%%%%%%%%%%%%%%%%%%%%%%%%%%%%%%%%%%%
\chapter{Reference for unit 'Retain'}
\label{computer:retain}
%%%%%%%%%%%%%%%%%%%%%%%%%%%%%%%%%%%%%%%%%%%%%%%%%%%%%%%%%%%%%%%%%%%%%%%
% Used units
\section{Used units}
\begin{FPCltable}{lr}{Used units by unit 'Retain'}{Retain:0units}
Name & Page \\ \hline
Classes\index{unit!Classes} & \pageref{computer:retain:classes} \\
Controls\index{unit!Controls} & \pageref{computer:retain:controls} \\
Dialogs\index{unit!Dialogs} & \pageref{computer:retain:dialogs} \\
Forms\index{unit!Forms} & \pageref{computer:retain:forms} \\
Graphics\index{unit!Graphics} & \pageref{computer:retain:graphics} \\
LResources\index{unit!LResources} & \pageref{computer:retain:lresources} \\
System\index{unit!System} & \pageref{computer:retain:system} \\
SysUtils\index{unit!SysUtils} & \pageref{computer:retain:sysutils} \\
\end{FPCltable}
%%%%%%%%%%%%%%%%%%%%%%%%%%%%%%%%%%%%%%%%%%%%%%%%%%%%%%%%%%%%%%%%%%%%%%%
% Procedures and functions
\section{Procedures and functions}
\label{retainfunctions}
% Register
\subsection{Register}
\label{computer:retain:register}
\index{Register}
\begin{FPCList}
\Declaration 

\begin{verbatim}
procedure Register
\end{verbatim}
\Visibility
default
\end{FPCList}
%%%%%%%%%%%%%%%%%%%%%%%%%%%%%%%%%%%%%%%%%%%%%%%%%%%%%%%%%%%%%%%%%%%%%%%
% TRetainData
\section{TRetainData}
\label{computer:retain:tretaindata}
\index{TRetainData}
% Method overview
\subsection{Method overview}
\label{computer:retain:tretaindata:methods}
\begin{tabularx}{\textwidth}{llX}
Page & Method & Description  \\ \hline
\pageref{computer:retain:tretaindata:beforedestruction} & BeforeDestruction  &  \\
\pageref{computer:retain:tretaindata:create} & Create  &  \\
\pageref{computer:retain:tretaindata:loadfromfile} & LoadFromFile  &  \\
\pageref{computer:retain:tretaindata:savetofile} & SaveToFile  &  \\
\pageref{computer:retain:tretaindata:update} & Update  &  \\
\hline
\end{tabularx}
% Property overview
\subsection{Property overview}
\label{computer:retain:tretaindata:properties}
\begin{tabularx}{\textwidth}{lllX}
Page & Properties & Access & Description \\ \hline
\pageref{computer:retain:tretaindata:haschanged} & HasChanged & rw &  \\
\hline
\end{tabularx}
% TRetainData.LoadFromFile
\subsection{TRetainData.LoadFromFile}
\label{computer:retain:tretaindata:loadfromfile}
\index{TRetainData.LoadFromFile}
\begin{FPCList}
\Declaration 

\begin{verbatim}
procedure LoadFromFile;  Virtual
\end{verbatim}
\Visibility
protected
\end{FPCList}
% TRetainData.SaveToFile
\subsection{TRetainData.SaveToFile}
\label{computer:retain:tretaindata:savetofile}
\index{TRetainData.SaveToFile}
\begin{FPCList}
\Declaration 

\begin{verbatim}
procedure SaveToFile;  Virtual
\end{verbatim}
\Visibility
protected
\end{FPCList}
% TRetainData.Create
\subsection{TRetainData.Create}
\label{computer:retain:tretaindata:create}
\index{TRetainData.Create}
\begin{FPCList}
\Declaration 

\begin{verbatim}
constructor Create(AOwner: TComponent);  Override
\end{verbatim}
\Visibility
public
\end{FPCList}
% TRetainData.BeforeDestruction
\subsection{TRetainData.BeforeDestruction}
\label{computer:retain:tretaindata:beforedestruction}
\index{TRetainData.BeforeDestruction}
\begin{FPCList}
\Declaration 

\begin{verbatim}
procedure BeforeDestruction;  Override
\end{verbatim}
\Visibility
public
\end{FPCList}
% TRetainData.Update
\subsection{TRetainData.Update}
\label{computer:retain:tretaindata:update}
\index{TRetainData.Update}
\begin{FPCList}
\Declaration 

\begin{verbatim}
procedure Update;  Virtual
\end{verbatim}
\Visibility
public
\end{FPCList}
% TRetainData.HasChanged
\subsection{TRetainData.HasChanged}
\label{computer:retain:tretaindata:haschanged}
\index{TRetainData.HasChanged}
\begin{FPCList}
\Declaration 

\begin{verbatim}
Property HasChanged : Boolean
\end{verbatim}
\Visibility
public
\Access
Read,Write
\end{FPCList}
%%%%%%%%%%%%%%%%%%%%%%%%%%%%%%%%%%%%%%%%%%%%%%%%%%%%%%%%%%%%%%%%%%%%%%%
%%%%%%%%%%%%%%%%%%%%%%%%%%%%%%%%%%%%%%%%%%%%%%%%%%%%%%%%%%%%%%%%%%%%%%%
% Reference for unit 'SysFS'
%%%%%%%%%%%%%%%%%%%%%%%%%%%%%%%%%%%%%%%%%%%%%%%%%%%%%%%%%%%%%%%%%%%%%%%
\chapter{Reference for unit 'SysFS'}
\label{computer:sysfs}
%%%%%%%%%%%%%%%%%%%%%%%%%%%%%%%%%%%%%%%%%%%%%%%%%%%%%%%%%%%%%%%%%%%%%%%
% Used units
\section{Used units}
\begin{FPCltable}{lr}{Used units by unit 'SysFS'}{SysFS:0units}
Name & Page \\ \hline
Classes\index{unit!Classes} & \pageref{computer:sysfs:classes} \\
System\index{unit!System} & \pageref{computer:sysfs:system} \\
SysUtils\index{unit!SysUtils} & \pageref{computer:sysfs:sysutils} \\
\end{FPCltable}
%%%%%%%%%%%%%%%%%%%%%%%%%%%%%%%%%%%%%%%%%%%%%%%%%%%%%%%%%%%%%%%%%%%%%%%
% Overview
\section{Overview}
Viele mit dem Computer verbundene Geräte werden auf das sysfs abgebildet. Dort kann man sie als Objektorientierte Struktur über Dateifunktionen ansprechen. So wird die Hardware und der Zugriff darauf abstrakt und Anwendungsprogrammierung unabhängig davon. Beispiel: Ein Gerät wird als Verzeichnis dargestellt. In diesem Verzeichnis befinden sich Dateien und Unterverzeichnisse, die sich zu dem Gerät(everzeichnis) wie Eigenschaften zu einem Objekt verhalten. Eigenschaften, die keine Verzeichnisse sind, sind Textdateien. Der darin enthaltene Text ist der Eigenschaftswert. Der Dateiname kann als Eigenschaftsname verstanden werden. Die Funktionen in dieser Unit vereinfachen den Zugriff auf solche Dateien, indem man ihnen den Dateinamen als Eigenschaftsname übergibt. Ein zu lesender Eigenschaftswert wird als Funktionsergebnis zurückgegeben. Ein zu schreibender Wert wird mit dem Dateinamen (als Eigenschaftsname) als weiterer Parameter übergeben. Alle erforderlichen Dateioperationen für einen Schreib- oder Lesevorgang und die Parametertypumwandlungen sind in der jeweiligen Funktion gekapselt.%%%%%%%%%%%%%%%%%%%%%%%%%%%%%%%%%%%%%%%%%%%%%%%%%%%%%%%%%%%%%%%%%%%%%%%
% Constants, types and variables
\section{Constants, types and variables}
\label{sysfsconststypesvars}
% Types
\subsection{Types}
\label{sysfstypes}

\begin{verbatim}
TGetIntFunc = function(AttrobuteFileName: string) : LongInt
\end{verbatim}
\label{computer:sysfs:tgetintfunc}
\index{TGetIntFunc}


%%%%%%%%%%%%%%%%%%%%%%%%%%%%%%%%%%%%%%%%%%%%%%%%%%%%%%%%%%%%%%%%%%%%%%%
% Procedures and functions
\section{Procedures and functions}
\label{sysfsfunctions}
% GetAttribute
\subsection{GetAttribute}
\label{computer:sysfs:getattribute}
\index{GetAttribute}
\begin{FPCList}
\Synopsis
Attributabfrage als Zeichenkette\Declaration 

\begin{verbatim}
function GetAttribute(AttributeFileName: string) : string
\end{verbatim}
\Visibility
default
\Description
fragt den Inhalt einer Datei als Zeichenkette ab\end{FPCList}
% GetBooleanAttribute
\subsection{GetBooleanAttribute}
\label{computer:sysfs:getbooleanattribute}
\index{GetBooleanAttribute}
\begin{FPCList}
\Synopsis
Attributabfrage als Zeichenkette\Declaration 

\begin{verbatim}
function GetBooleanAttribute(AttributeFileName: string) : Boolean
\end{verbatim}
\Visibility
default
\Description
fragt den Inhalt einer Datei als Wahrheitswert ab\end{FPCList}
% GetIntAttribute
\subsection{GetIntAttribute}
\label{computer:sysfs:getintattribute}
\index{GetIntAttribute}
\begin{FPCList}
\Declaration 

\begin{verbatim}
function GetIntAttribute(AttributeFileName: string) : LongInt
\end{verbatim}
\Visibility
default
\end{FPCList}
% SetAttribute
\subsection{SetAttribute}
\label{computer:sysfs:setattribute}
\index{SetAttribute}
\begin{FPCList}
\Declaration 

\begin{verbatim}
procedure SetAttribute(AttributeFileName: string; Value: LongInt)
                      ;  Overload
procedure SetAttribute(AttributeFileName: string; Value: string)
                      ;  Overload
procedure SetAttribute(AttributeFileName: string; Value: Boolean)
                      ;  Overload
\end{verbatim}
\Visibility
default
\end{FPCList}
